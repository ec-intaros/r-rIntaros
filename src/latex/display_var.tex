\inputencoding{latin1}
\HeaderA{display\_var}{Display the IMR variable}{display.Rul.var}
\keyword{classes}{display\_var}
%
\begin{Description}\relax
Display the IMR variable
\end{Description}
%
\begin{Usage}
\begin{verbatim}
display_var(dbin, var = NA, var_scale = NA, title = NA, flag.mesh = TRUE,
            mesh = 1, flag.xvalid = FALSE, flag.coast = TRUE,
	    colors = rg.colors(), filename = NA, ...)
\end{verbatim}
\end{Usage}
%
\begin{Arguments}
\begin{ldescription}
\item[\code{dbin}] 
The \code{\LinkA{db-class}{db.Rdash.class}} structure organized as a regular grid.

\item[\code{var}] 
Name of the Target Variable

\item[\code{var\_scale}] 
Range of the Color scale

\item[\code{title}] 
Title attached to the figure

\item[\code{flag.mesh}] 
When TRUE, the grid mesh is overlaid

\item[\code{mesh}] 
Mesh of the grid to be overlaid (only if 'flag.mesh'=TRUE)

\item[\code{flag.xvalid}] 
When TRUE, the target variable represents the standardized cross-validation
error. A colored symbol is posted representing its absolute value.
Otherwise, the colored symbol is proportional to its value.

\item[\code{flag.coast}] 
When TRUE, the coast line is overlaid

\item[\code{colors}] 
Set of colors used for graphic representation

\item[\code{filename}] 
Name of the PNG file where the figure is saved. Tis is used only if
the flag\_file has been defined in the Intaros Environment.
The resulting file is then stored in the Directory defined in the
Intaros Environment.

\item[\code{...}] 
Argument passed to \code{\LinkA{display\_grid\_mesh}{display.Rul.grid.Rul.mesh}} and
\code{\LinkA{db.plot}{db.plot}}.

\end{ldescription}
\end{Arguments}
%
\begin{Value}
Unused
\end{Value}
