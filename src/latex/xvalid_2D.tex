\inputencoding{latin1}
\HeaderA{xvalid\_2D}{2-D Cross-Validation}{xvalid.Rul.2D}
\keyword{classes}{xvalid\_2D}
%
\begin{Description}\relax
2-D Cross-Validation
\end{Description}
%
\begin{Usage}
\begin{verbatim}
xvalid_2D(dbin, var, vario_lag = 0.5, vario_nlag = 20,
          moving = FALSE, nmaxi = 40, model = NA, struct = c(1, 12),
	  dirvect = c(0, 45, 90, 135), radix = "Xvalid", draw.model = FALSE,
	  verbose = FALSE, ...)
\end{verbatim}
\end{Usage}
%
\begin{Arguments}
\begin{ldescription}
\item[\code{dbin}] 
The \code{\LinkA{db-class}{db.Rdash.class}} structure containing the IMR data.

\item[\code{var}] 
Name of the Target variable

\item[\code{vario\_lag}] 
Lag of the experimental variogram calculated

\item[\code{vario\_nlag}] 
Number of variogram lags to be calculated

\item[\code{moving}] 
When TRUE, a Moving Neighborhood is used. Otherwise the Neighborhood is
Unique.

\item[\code{nmaxi}] 
Maximum number of samples used per Neighborhood. This parameter is
used only if the flag 'moving' is set to TRUE.

\item[\code{model}] 
The \code{\LinkA{model-class}{model.Rdash.class}} used for interpolation. When defined,
the steps for calculating the Experimental Variogram and fitting the Model
are not performed.

\item[\code{struct}] 
Set of basic structures used for fitting the Model

\item[\code{dirvect}] 
Set of directions where the (horizontal) experimental variograms must be
calculated. If not defined, an omni-directional is calculated instead.

\item[\code{radix}] 
Radix attached to the resulting variables

\item[\code{draw.model}] 
When TRUE, the experimental variogram and the fitted model are represented
graphically.

\item[\code{verbose}] 
Verbose flag

\item[\code{...}] 
Arguments passed to \code{\LinkA{prepar\_model}{prepar.Rul.model}}.

\end{ldescription}
\end{Arguments}
%
\begin{Value}
The \code{\LinkA{db-class}{db.Rdash.class}} which corresponds to the input Db, to which the
cross-validation results are added.
\end{Value}
