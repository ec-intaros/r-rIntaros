\inputencoding{latin1}
\HeaderA{interpolate\_3D}{Interpolate Target Variable on a 3-D grid}{interpolate.Rul.3D}
\keyword{classes}{interpolate\_3D}
%
\begin{Description}\relax
Interpolate Target Variable on a 3-D grid
\end{Description}
%
\begin{Usage}
\begin{verbatim}
interpolate_3D(dbin, var, mesh = 1, depth0, ndepth, ddepth,
               vario_lag = 0.5, vario_nlag = 20, struct = c(1, 12),
	       dirvect = c(0, 45, 90, 135), draw.model = FALSE,
	       verbose = FALSE,...)
\end{verbatim}
\end{Usage}
%
\begin{Arguments}
\begin{ldescription}
\item[\code{dbin}] 
The \code{\LinkA{db-class}{db.Rdash.class}} structure containing the IMR data.

\item[\code{var}] 
Name of the Target variable

\item[\code{mesh}] 
Horizontal Mesh of the resulting grid

\item[\code{depth0}] 
Origin of the 3-D grid along Depth

\item[\code{ndepth}] 
Number of meshes of the 3-D grid along Depth

\item[\code{ddepth}] 
Mesh of the 3-D grid along Depth

\item[\code{vario\_lag}] 
Lag of the experimental variogram calculated

\item[\code{vario\_nlag}] 
Number of variogram lags to be calculated

\item[\code{struct}] 
Set of basic structures used for fitting the Model

\item[\code{dirvect}] 
Set of directions where the (horizontal) experimental variograms must be
calculated. If not defined, an omni-directional is calculated instead.

\item[\code{draw.model}] 
When TRUE, the experimental variogram and the fitted model are represented
graphically.

\item[\code{verbose}] 
Verbose flag

\item[\code{...}] 
Arguments passed to \code{\LinkA{prepar\_model}{prepar.Rul.model}}.

\end{ldescription}
\end{Arguments}
%
\begin{Value}
A new \code{\LinkA{db-class}{db.Rdash.class}} organized as a regular 3-D grid which contains
the estimation result as well as the variance of the estimation error.
\end{Value}
