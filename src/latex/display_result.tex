\inputencoding{latin1}
\HeaderA{display\_result}{Display the results of an interpolation}{display.Rul.result}
\keyword{classes}{display\_result}
%
\begin{Description}\relax
Display the results of an interpolation
\end{Description}
%
\begin{Usage}
\begin{verbatim}
display_result(dbin = NA, dbg, var = NA, depth, var_scale = NA,
               flag.estim = TRUE, flag.coast = TRUE,
	       colors = rg.colors(), filename = NA, ...)
\end{verbatim}
\end{Usage}
%
\begin{Arguments}
\begin{ldescription}
\item[\code{dbin}] 
The \code{\LinkA{db-class}{db.Rdash.class}} structure organized containing the IMR
information.

\item[\code{dbg}] 
The \code{\LinkA{db-class}{db.Rdash.class}} structure organized as a regular grid which
contains the estimation results

\item[\code{var}] 
Name of the Target Variable

\item[\code{depth}] 
Value of the estimation depth (used for building the title)

\item[\code{var\_scale}] 
Limits defined for the color scale attached to the Target Variable

\item[\code{flag.estim}] 
When TRUE, the estimation variable is displayed.
When FALSE, the standard deviation of the estimation error is displayed.

\item[\code{flag.coast}] 
When TRUE, the coast line is overlaid

\item[\code{colors}] 
Set of colors used for graphic representation.

\item[\code{filename}] 
Name of the PNG file where the figure is saved. Tis is used only if
the flag\_file has been defined in the Intaros Environment.
The resulting file is then stored in the Directory defined in the
Intaros Environment.

\item[\code{...}] 
Argument passed to \code{\LinkA{db.plot}{db.plot}}.

\end{ldescription}
\end{Arguments}
%
\begin{Value}
Unused
\end{Value}
