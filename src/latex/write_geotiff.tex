\inputencoding{latin1}
\HeaderA{write\_geotiff}{Write a Geotiff file for each variable in the Db with locator "z"}{write.Rul.geotiff}
\keyword{geotiff}{write\_geotiff}
%
\begin{Description}\relax
Write a Geotiff file (colorized or mono band) for each interest variable from the grid Db with locator "z" into the output image directory defined in the environment
\end{Description}
%
\begin{Usage}
\begin{verbatim}
write_geotiff(dbg, colors=NA, colors.files=NA, prefix=NA)
\end{verbatim}
\end{Usage}
%
\begin{Arguments}
\begin{ldescription}
\item[\code{dbg}] 
The \code{\LinkA{db-class}{db.Rdash.class}} structure organized as a regular grid.

\item[\code{colors}] 
The colors lists to be used (one by variable). First color of each colorscale will be affected to corresponding minimal value and last color for maximal value. A colorscale is a liste of text colors (hexadecimal or plain text). It can be for example the colorscale returned by \code{\LinkA{rg.colors}{rg.colors}}. Not used if \code{colors.file} is defined.

\item[\code{colors.files}] 
List of colorscale text pre-existing files to be used (one by variable).

\item[\code{Prefix}] 
Prefix to be append to the front of each Geotiff file name.

\end{ldescription}
\end{Arguments}
%
\begin{Value}
Unused
\end{Value}
