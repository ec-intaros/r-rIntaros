\inputencoding{latin1}
\HeaderA{prepar\_vario}{Caculate Experimental Variogram}{prepar.Rul.vario}
\keyword{classes}{prepar\_vario}
%
\begin{Description}\relax
Caculate Experimental Variogram
\end{Description}
%
\begin{Usage}
\begin{verbatim}
prepar_vario(dbin, dirvect = c(0, 45, 90, 135),
             vario_lag = 0.5, vario_nlag = 20, bench = 0, nbench = 0,
	     draw.vario = FALSE, verbose = FALSE, ...)
\end{verbatim}
\end{Usage}
%
\begin{Arguments}
\begin{ldescription}
\item[\code{dbin}] 
The \code{\LinkA{db-class}{db.Rdash.class}} structure containing the IMR data.

\item[\code{dirvect}] 
Set of directions where the (horizontal) experimental variograms must be
calculated. If not defined, an omni-directional is calculated instead.

\item[\code{vario\_lag}] 
Lag of the experimental variogram calculated

\item[\code{vario\_nlag}] 
Number of variogram lags to be calculated

\item[\code{bench}] 
Slicing width (along Depth) used for Variogram calculation

\item[\code{nbench}] 
Number of vertical slices (along Depth) used for calculating the
variogram along Depth axis

\item[\code{draw.vario}] 
When TRUE, the experimental variogram is represented

\item[\code{verbose}] 
Verbose flag

\item[\code{...}] 
Argument passed to \code{\LinkA{vario.plot}{vario.plot}}.

\end{ldescription}
\end{Arguments}
%
\begin{Value}
A \code{\LinkA{vario-class}{vario.Rdash.class}} structure containing the resulting Experimental
Variogram.
\end{Value}
